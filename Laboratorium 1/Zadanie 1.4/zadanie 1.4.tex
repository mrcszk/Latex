\documentclass[a4paper,11pt]{article}
\usepackage[english]{babel}
\usepackage[utf8]{inputenc}   % lub utf8
\usepackage[T1]{fontenc}
\usepackage{graphicx}
\usepackage{anysize}
\usepackage{enumerate}
\usepackage{times}
\usepackage{amssymb}
\usepackage{amsthm}

%\marginsize{left}{right}{top}{bottom}
\marginsize{3cm}{3cm}{3cm}{3cm}
\sloppy

\begin{document}
	\setcounter{section}{1}
	\setcounter{subsection}{2}
	\subsubsection{The Model-Checking Process}
	In applying model checking to a design the 
	following different phases can be distinguished:
	\begin{itemize}
	\item \textsl{Modeling} phase: 
	\begin{enumerate}[--]
		\item model the system under consideration using the model description language of the model checker at hand; 
		\item as a first sanity check and quick assessment of the model perform some simulations; 
		\item formalize the property to be checked using the property specification language.
	\end{enumerate} 
	\item \textsl{Running} phase: run the model checker to check the validity of the property in the system
		model. 
		\item \textsl{Analysis} phase: 
	\begin{enumerate}[--]
		\item property satisfied? $\rightarrow$ check next property (if any); 
		\item property violated? $\rightarrow$
		\begin{enumerate}[1.]
			\item analyze generated counterexample by simulation; 
			\item refine the model, design, or property; 
			\item repeat the entire procedure. 
		\end{enumerate} 
		\item out of memory? $\rightarrow$ try to reduce the model and try again.
		
	\end{enumerate}
\end{itemize}
\end{document}