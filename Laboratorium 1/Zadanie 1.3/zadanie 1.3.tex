\documentclass[a4paper,11pt]{article}
\usepackage[polish]{babel}
\usepackage[utf8]{inputenc}
\usepackage[T1]{fontenc}
\usepackage{graphicx}
\usepackage{anysize}
\usepackage{enumerate}
\usepackage{times}
\usepackage{amsthm}
\usepackage{minipage-marginpar}
 
%\marginsize{left}{right}{top}{bottom}
\marginsize{3cm}{3cm}{3cm}{3cm}
\sloppy
\newcount\spk
\def\beginscript {\bgroup \parindent=0pt \spk=1 \sl \rightskip,4in
	\def\par {\ifnum\spk=1 \endgraf \it \spk=2 \leftskip,4in \rightskip0in
		\else \endgraf \sl \spk=1 \leftskip0in \rightskip,4in \fi}}

\theoremstyle{definition}
\newtheorem{tw}{Warstwa}
\begin{document}
 
\section*{Komunikacja w sieci}
 W modelu ISO-OSI (Open System Interconnection Reference Model) całą procedurę przesyłania komunikatu podzielono na siedem warstw zajmujących się odrębnymi zagadnieniami. W każdej warstwie obowiązują szczegółowe zasady wymiany informacji, zwane protokołem.
 \begin{table}[htbp]
 \begin{center}
 \begin{tabular}{|l|c|}
 	\hline
 	Nazwa warstwy& Numer warstwy \\
 	\hline \hline
 	aplikacji&7\\\hline
 	prezentacji&6\\\hline
 	sesji&5\\\hline\hline
 	transportu&4\\\hline
 	sieci&3\\\hline
 	łącza danych&2\\\hline
 	fizyczna&1\\\hline
 	
 \end{tabular}
\end{center}
 \end{table}

Dzięki warstwowej strukturze model OSI jest bardzo elastyczny i daje się stosować do komunikacji zarówno w sieciach lokalnych, jak i rozległych. Podział na warstwy zwiększa jednak czas przesłania komunikatu i wydłuża go, gdyż każda warstwa dodaje własne informacje. Dlatego w szybkich sieciach lokalnych najniższe trzy warstwy zlewa się w jedną. Nie przeszkadza to komunikowaniu się tym sieciom z innymi sieciami na wyższych poziomach.

\begin{tw}\textbf{: FIZYCZNA}
\hangindent=0.4 true in \hangafter=1
 -- Jest ona odpowiedzialna za przesyłanie strumieni bitów. Odbiera ramki danych z warstwy 2 i przesyła szeregowo, bit po bicie, całą ich strukturę oraz zawartość. Jest ona również odpowiedzialna za odbiór kolejnych bitów przychodzących strumieni danych. Strumienie te są następnie przesyłane do warstwy łącza danych w celu ich ponownego ukształtowania.
\end{tw}
\begin{tw}\textbf{: ŁĄCZA DANYCH}
	\hangindent=0.4 true in \hangafter=1
	-- Jest ona odpowiedzialna za koncowa˛ zgodność przesyłania
	danych. W zakresie zadan zwia˛zanych z przesyłaniem, warstwa ła˛cza danych jest odpowiedzialna
	za upakowanie instrukcji, danych itp. w tzw. ramki. Ramka jest struktura˛ właściwa˛ dla warstwy ła˛cza danych, która zawiera ilość informacji wystarczaja˛ca˛ do pomyslnego
	przesyłania danych przez siec lokalna˛ do ich miejsca docelowego. Pomyslna transmisja
	danych zachodzi wtedy, gdy dane osia˛gaja˛ miejsce docelowe w postaci niezmienionej
	w stosunku do postaci, w której zostały wysłane. Ramka musi wi˛ec zawierac mechanizm
	umoz˙liwiaja˛cy weryfikowanie integralnosci jej zawartosci podczas transmisji. W wielu sytuacjach
	wysyłane ramki moga˛ nie osia˛gna˛c miejsca docelowego lub ulec uszkodzeniu podczas
	transmisji. Warstwa ła˛cza danych jest odpowiedzialna za rozpoznawanie i naprawe˛
	kaz˙dego takiego błe˛du. Warstwa ła˛cza danych jest równiez˙ odpowiedzialna za ponowne
	składanie otrzymanych z warstwy fizycznej strumieni binarnych i umieszczanie ich w ramkach.
	Ze wzgl˛edu na fakt przesyłania zarówno struktury, jak i zawartosci ramki, warstwa
	ła˛cza danych nie tworzy ramek od nowa. Buforuje ona przychodza˛ce bity dopóki nie uzbiera
	w ten sposób całej ramki.
\end{tw}

\begin{tw}\textbf{: SIECI}
	\hangindent=0.4 true in \hangafter=1
	-- Warstwa sieci jest odpowiedzialna za okreslenie trasy transmisji mi˛edzy
	komputerem-nadawca, a komputerem-odbiorca˛. Warstwa ta nie ma zadnych wbudowanych mechanizmów korekcji błe˛dów i w zwia˛zku z tym musi polegac na wiarygodnej transmisji
	koncowej warstwy ła˛cza danych. Warstwa sieci uz˙ywana jest do komunikowania sie˛
	z komputerami znajduja˛cymi sie˛ poza lokalnym segmentem sieci LAN. Umoz˙liwia im to
	własna architektura trasowania, niezale˙zna od adresowania fizycznego warstwy 2. Korzystanie
	z warstwy sieci nie jest obowia˛zkowe. Wymagane jest jedynie wtedy, gdy komputery
	komunikuja˛ce sie˛ znajduja˛ sie˛ w róz˙nych segmentach sieci przedzielonych routerem.\medskip
	
	\noindent \hspace{0.9cm}
	Najbardziej znanym protokołem warstwy sieci jest protokół IP (Internet Protocol) obowia˛-zuja˛cy w sieci Internet. Dzieli on przekazywany komunikat na odpowiedniej wielkosci
	(64 KB) pakiety i przesyła je od komputera do komputera w kierunku komputera docelowego.
	IP nie gwarantuje dostarczenia pakietu na miejsce. Nie sprawdza on równie˙z, czy
	pakiet, który dotarł ju˙z do celu, nie został czasem przekłamany. Docieraniem pakietów na
	miejsce i ich poprawnoscia˛ musi sie˛ zajmowac wyzsza warstwa transportu. IP współpracuje
	z wieloma (do 256) protokołami warstwy transportu (takimi jak TCP, UDP czy ICMP).
	Ka˙zdy pakiet ma w swym nagłówku informacj˛e o tym, którego typu protokołu transportu
	dotyczy.
\end{tw}
\begin{tw}\textbf{: TRANSPORTU}
	\hangindent=0.4 true in \hangafter=1
	-- Warstwa ta pełni funkcje˛ podobna˛ do funkcji warstwy ła˛cza
	w tym sensie, ze jest odpowiedzialna za koncowa˛ integralność transmisji. Jednak w odróz
	˙nieniu od warstwy ła˛cza danych – warstwa transportu umoz˙liwia te˛ usługe˛ równiez˙ poza
	lokalnymi segmentami sieci LAN. Potrafi bowiem wykrywac pakiety, które zostały przez
	routery odrzucone i automatycznie generowac z˙a˛danie ich ponownej transmisji. Warstwa
	transportu identyfikuje oryginalna˛ sekwencje˛ pakietów i ustawia je w oryginalnej kolejno-
	sci przed wysłaniem ich zawartosci do warstwy sesji.\medskip
	
	\noindent \hspace{0.9cm}
	TCP (Transmission Control Protocol) jest najbardziej znanym protokołem warstwy transportu.
	Poła˛czenie w TCP jest nawia˛zywane przez trzykrotne „podanie sobie re˛ki”. Niezawodność przesyłania danych jest osia˛gnie˛ta dzie˛ki numerowaniu pakietów, stosowaniu potwierdzen, ponownej transmisji, jesli nie było potwierdzenia przez zbyt długi czas. W celu
	zwie˛kszenia przepustowosci TCP stosuje tzw. metode˛ „przesuwaja˛cych sie˛ okienek”, która
	umo˙zliwia wysyłanie kilku pakietów bez czekania na ich potwierdzenie.
\end{tw}
 \begin{tw}\textbf{: SESJI}
 	\hangindent=0.4 true in \hangafter=1
 	-- Jest ona rzadko uz˙ywana; wiele protokołów funkcje tej warstwy doła˛cza
 	do swoich warstw transportowych. Zadaniem warstwy sesji jest zarza˛dzanie przebiegiem
 	komunikacji podczas poła˛czenia mie˛dzy dwoma komputerami. Przepływ tej komunikacji
 	nazywany jest sesja˛. Sesja moze słuzyc do doła˛czenia uzytkownika do odległego systemu,
 	albo do przesyłania zbiorów mi˛edzy róznymi maszynami (np. polecenie ftp). Jesli warstwa
 	transportu jest zawodna, zadaniem warstwy sesji jest tez˙ ponowne nawia˛zanie poła˛-
 	czenia w przypadku jego przerwania. Warstwa ta okresla, czy komunikacja moze zachodzic
 	w jednym, czy obu kierunkach. 	
 	Gwarantuje równiez zakonczenie wykonywania bieza˛cego
 	z˙a˛dania przed przyje˛ciem kolejnego.\medskip
 	
 	\noindent \hspace{0.9cm} 	
 	Jednym z najbardziej popularnych protokołów warstwy sieci jest protokół RPC (Remote
 	Procedure Call – zdalne wywołanie procedury). Protokół ten zajmuje si˛e wysyłaniem przez
 	siec zadan od klientów do serwerów i odbieraniem odpowiedzi. RPC musi umiec zlokalizowac komputer, na którym wykonuje si˛e serwer, reagowac w przypadku, gdy serwera nie ma
 	oraz rejestrowac pojawienie si˛e nowych serwerów. Poniewa˙z program serwera mo˙ze byc wykonywany na komputerze o zupełnie innej architekturze ni˙z architektura komputera, na
 	którym jest wykonywany program klienta, protokół RPC musi zadbac o odpowiednie przekształcenie
 	przesyłanych danych. Jesli odpowiedz na wysłane za˛danie nie nadchodzi zbyt
 	długo, RPC ponawia wysłanie za˛dania. Musi przy tym zadbac, by to ponowione z˙a˛danie nie
 	zostało zrozumiane jako zupełnie nowe. Za pomoca˛protokołu RPC moz˙na takze realizowac
 	rozgłaszanie, czyli wysłanie za˛dania jednoczesnie do wielu serwerów. Klient ma wówczas
 	kilka mo˙zliwosci: moze czekac na reakcje od wszystkich serwerów, gdy do dalszej pracy
 	potrzebuje wszystkich usług; moze czekac tylko na jeden serwer, jesli wysłał komunikat
 	typu „niech mi to ktos zrobi”; mo˙ze tez nie czekac w ogóle, jesli celem było jedynie poinformowanie
 	o czyms serwerów. Protokół RPC jest ogólnie uznana˛metoda˛ komunikowania
 	si˛e w systemach typu klient-serwer.
 \end{tw}
\begin{tw}\textbf{: PREZENTACJI}
	\hangindent=0.4 true in \hangafter=1
	-- Warstwa prezentacji jest odpowiedzialna za zarza˛dzanie sposobem
	kodowania wszelkich danych. Nie ka˙zdy komputer korzysta z tych samych schematów
	kodowania danych, wi˛ec warstwa prezentacji odpowiedzialna jest za translacj˛e mi˛edzy niezgodnymi
	schematami kodowania danych. Warstwa ta moze byc równiez wykorzystywana
	do niwelowania ró˙znic mi˛edzy formatami zmiennopozycyjnymi, jak równie˙z do szyfrowania
	i rozszyfrowywania wiadomosci.\medskip
	
	\noindent \hspace{0.9cm} 
	W asymetrycznym systemie szyfrowania znajomość funkcji szyfrujacej nie wystarcza do
	odgadnie˛cia funkcji rozszyfrowuja˛cej. Funkcja szyfruja˛ca $S$ i funkcja deszyfruja˛ca $D$ maja˛
	taka˛ własnosć, z˙e dla kaz˙dego komunikatu $K, D(S(K)) = K$. Rozwia˛zanie problemu
	elektronicznych podpisów stało sie˛ moz˙liwe dzie˛ki wynalezieniu funkcji, które maja˛ takz˙e
	własnosc odwrotna˛ $S(D(K)) = K$. W kryptosystemie asymetrycznym kaz˙da ze stron ma
	dwa klucze: publiczny do szyfrowania i tajny do odszyfrowywania. Zaszyfrowac i wysłac
	komunikat moze zatem kaz˙dy, ale odczytac go potrafi tylko adresat. Informacja˛ o kluczach
	powinien zarza˛dzac specjalny proces-centrala, którego klucz publiczny jest jedynym ogólnie
	dost˛epnym. Aby zdobyc informacj˛e o kluczu osoby X, wysyła si˛e zapytanie do centrali
	(nieszyfrowane), a otrzymuje sie˛ zaszyfrowana˛ odpowiedz, która˛moz˙na odszyfrowac kluczem
	publicznym.
\end{tw}
\begin{tw}\textbf{: APLIKACJI}
	\hangindent=0.4 true in \hangafter=1
	-- Pełni ona role interfejsu pomi˛edzy aplikacjami u˙zytkownika a usługami	sieci. Warstwe˛ te˛ mozna uwazac za inicjuja˛ca˛ sesje komunikacyjne. Protokóły warstwy
	aplikacji to np.: HTTP, HTTPS, FTP, SSH, Telnet, POP3, SMTP.
\end{tw}
\end{document}