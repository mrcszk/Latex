\documentclass[a4paper,11pt]{book}
\usepackage[polish]{babel}
\usepackage[utf8]{inputenc}   % lub utf8
\usepackage[T1]{fontenc}
\usepackage{graphicx}
\usepackage{anysize}
\usepackage{enumerate}
\usepackage{times}

%\marginsize{left}{right}{top}{bottom}
\marginsize{3cm}{3cm}{3cm}{3cm}
\sloppy

\begin{document}
Z każdym działającym systemem komputerowym powiązane jest oczekiwanie 
{\em poprawności} jego działania (\cite{Sommerville:2010:SE:1841764}). Istnieje szeroka 
klasa systemów, dla których poprawność powiązana jest nie tylko z 
wynikami ich pracy, ale również z~czasem, w~jakim wyniki te są 
otrzymywane. Systemy takie nazywane są {\em systemami czasu 
rzeczywistego}, a~ponieważ są one rozpatrywane  w~kontekście swojego 
otoczenia, często określane są terminem {\em systemy wbudowane} 
(\cite{Sommerville:2010:SE:1841764}, \cite{Szmuc:Szpyrka:i:inni}). 

Ze względu na specyficzne cechy takich systemów, weryfikacja jakości 
tworzonego oprogramowania oparta wyłącznie na jego testach jest 
niewystarczająca. Coraz częściej w~takich sytuacjach, weryfikacja 
poprawności tworzonego systemu lub najbardziej istotnych jego 
modułów prowadzona jest z~zastosowaniem metod formalnych 
(\cite{Alur:1990:AMR:90397.90438}, ~\cite{Szmuc:Szpyrka:i:inni}). 

\bibliographystyle{unsrt}
\bibliography{zadanie_3.1}

\end{document}