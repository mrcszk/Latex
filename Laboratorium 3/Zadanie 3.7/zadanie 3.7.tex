\documentclass[a4paper,10pt]{report}
\usepackage[polish]{babel}
\usepackage[utf8]{inputenc}
\usepackage[T1]{fontenc}
\usepackage{times}
\usepackage{graphicx}
\usepackage{anysize}
\usepackage{longtable}
\usepackage{array}

%\marginsize{left}{right}{top}{bottom}
\marginsize{5cm}{5cm}{5cm}{5cm}

\begin{document}
	
	\begin{longtable}{|c|c|c|c|c|c|c|c|c|c|}
		\caption{Warunki terenowe}
		\label{tab:warunkiTerenowe}\\
		\cline{2-10} 
		\multicolumn{1}{l|}{} & \multicolumn{9}{|c|}{Zwrotnice} \\ \hline
		{Przebiegi} & 3/4 & 5 & 6 & 7/8 & 15/16 & 17 & 18 & 19/20 & 21/22 \\ \hline
		\endfirsthead
		\cline{2-10} 
		\multicolumn{1}{l|}{} & \multicolumn{9}{|c|}{Zwrotnice} \\ \hline
		{Przebiegi} & 3/4 & 5 & 6 & 7/8 & 15/16 & 17 & 18 & 19/20 & 21/22 \\ \hline
		\endhead
		\multicolumn{10}{|c|}{Stopka tabeli} \\ \hline
		\endfoot
		\multicolumn{2}{|c|c c c}{Identyfikator}&{ Liczba pudełek}&{ Wartość zamówienia i Termin realizacji} \\ \hline
		\endlastfoot
		B1 & + & + &  &  &  &  &  &  &  \\ \hline
		B2 & -- &  & + & o+ &  &  &  &  &  \\ \hline
		B3 & + & -- &  &  &  &  &  &  &  \\ \hline
		B4 & -- &  & -- & + & o+ &  &  &  &  \\ \hline
		R2 &  &  &  & o+ & o+ & + &  & + & + \\ \hline
		R4 &  &  &  & o+ & + & -- &  & + & + \\ \hline
		F2W & + &  & + & o+ &  &  &  &  &  \\ \hline
		G2W & + &  & -- & + &  &  &  &  &  \\ \hline
		K1D &  &  &  &  & + & -- &  & -- & + \\ \hline
		L1D &  &  &  &  & o+ & + &  & -- & + \\ \hline
		M1D &  &  &  &  &  &  & + & + & + \\ \hline
		N1D &  &  &  &  &  &  & -- & + & + \\ \hline
		B1 & + & + &  &  &  &  &  &  &  \\ \hline
		B2 & -- &  & + & o+ &  &  &  &  &  \\ \hline
		B3 & + & -- &  &  &  &  &  &  &  \\ \hline
		B4 & -- &  & -- & + & o+ &  &  &  &  \\ \hline
		R2 &  &  &  & o+ & o+ & + &  & + & + \\ \hline
		R4 &  &  &  & o+ & + & -- &  & + & + \\ \hline
		F2W & + &  & + & o+ &  &  &  &  &  \\ \hline
		G2W & + &  & -- & + &  &  &  &  &  \\ \hline
		K1D &  &  &  &  & + & -- &  & -- & + \\ \hline
		L1D &  &  &  &  & o+ & + &  & -- & + \\ \hline
		M1D &  &  &  &  &  &  & + & + & + \\ \hline
		N1D &  &  &  &  &  &  & -- & + & + \\ \hline
		B1 & + & + &  &  &  &  &  &  &  \\ \hline
		B2 & -- &  & + & o+ &  &  &  &  &  \\ \hline
		B3 & + & -- &  &  &  &  &  &  &  \\ \hline
		B4 & -- &  & -- & + & o+ &  &  &  &  \\ \hline
		R2 &  &  &  & o+ & o+ & + &  & + & + \\ \hline
		R4 &  &  &  & o+ & + & -- &  & + & + \\ \hline
		F2W & + &  & + & o+ &  &  &  &  &  \\ \hline
		G2W & + &  & -- & + &  &  &  &  &  \\ \hline
		K1D &  &  &  &  & + & -- &  & -- & + \\ \hline
		L1D &  &  &  &  & o+ & + &  & -- & + \\ \hline
		M1D &  &  &  &  &  &  & + & + & + \\ \hline
		N1D &  &  &  &  &  &  & -- & + & + \\ \hline
		B1 & + & + &  &  &  &  &  &  &  \\ \hline
		B2 & -- &  & + & o+ &  &  &  &  &  \\ \hline
		B3 & + & -- &  &  &  &  &  &  &  \\ \hline
		B4 & -- &  & -- & + & o+ &  &  &  &  \\ \hline
		R2 &  &  &  & o+ & o+ & + &  & + & + \\ \hline
		R4 &  &  &  & o+ & + & -- &  & + & + \\ \hline
		F2W & + &  & + & o+ &  &  &  &  &  \\ \hline
		G2W & + &  & -- & + &  &  &  &  &  \\ \hline
		K1D &  &  &  &  & + & -- &  & -- & + \\ \hline
		L1D &  &  &  &  & o+ & + &  & -- & + \\ \hline
		M1D &  &  &  &  &  &  & + & + & + \\ \hline
		N1D &  &  &  &  &  &  & -- & + & + \\ \hline
		B1 & + & + &  &  &  &  &  &  &  \\ \hline
		B2 & -- &  & + & o+ &  &  &  &  &  \\ \hline
		B3 & + & -- &  &  &  &  &  &  &  \\ \hline
		B4 & -- &  & -- & + & o+ &  &  &  &  \\ \hline
		R2 &  &  &  & o+ & o+ & + &  & + & + \\ \hline
		R4 &  &  &  & o+ & + & -- &  & + & + \\ \hline
		F2W & + &  & + & o+ &  &  &  &  &  \\ \hline
		G2W & + &  & -- & + &  &  &  &  &  \\ \hline
		K1D &  &  &  &  & + & -- &  & -- & + \\ \hline
		L1D &  &  &  &  & o+ & + &  & -- & + \\ \hline
		M1D &  &  &  &  &  &  & + & + & + \\ \hline
		N1D &  &  &  &  &  &  & -- & + & + \\ \hline
	\end{longtable}
	
	Pospolity ptak lęgowy i~przelotny. Przylatuje w~połowie kwietnia, odlatuje od sierpnia do września, czasem dopiero w~październiku. Żyje w~mniejszych miastach i~na wsiach, unika większych miast. Gnieździ się w~budynkach, najchętniej w~stajniach i~oborach, pod dachem. Gniazdo ma kształt miskowaty, zbudowane jest z~gliny i~źdźbeł trawy zmieszanych ze śliną. Wnętrze wysłane jest pierzem i~sierścią. Gatunek monogamiczny. Gniazduje dwa razy w roku -- w~maju i~w~czerwcu. Znosi 4-6 białych jaj kształtu jajowatego z~fioletowoszarymi i~brązowoczerwonymi plamkami. Wysiadywanie trwa 14-16 dni. Pisklęta są rzekomymi gniazdownikami, legną się okryte białawym puchem. Wykarmianie piskląt trwa około 24 dni. 
	
	

	
	Liczny gatunek lęgowy i~przelotny. Przylatuje w~kwietniu, odlatuje we wrześniu. Przebywa w~miejscach, w~których żyje również jaskółka dymówka. Gniazdo buduje pod dachami budynków, rzadziej gnieździ się wewnątrz budynku. Gniazdo jest ulepione z~gliny zmieszanej z~błotem i~śliną, w~kształcie ćwierć kuli przyczepionej pod okapem dachu do ściany. Wewnątrz wysłane jest pierzem i~sierścią. Gatunek monogamiczny. Gniazduje dwa razy w~ciągu roku. Znosi 4-5 czystobiałych jaj, kształtu jajowatego. Wysiaduje samiczka i~samiec przez około dwa tygodnie. Pisklęta karmione przez 21-23 dni, są rzekomymi gniazdownikami, lęgną się okryte białawym puchem. Na rysunku~\ref{fig:jaskolkaoknowka} pokazano jaskółkę oknówkę.
	
	\begin{table}
		\centerline{
			\begin{tabular}{|p{8mm}|p{8mm}|p{8mm}|}
				\hline
				x y z x y z x y z x y z x y z & x y z x y z x y z & 1 1 1 1 1 \\\hline
		\end{tabular}}
	\end{table}
	
	\begin{table}
		\centerline{
			\begin{tabular}{|m{8mm}|m{8mm}|m{8mm}|}
				\hline
				x y z x y z x y z x y z x y z & x y z x y z x y z & 1 1 1 1 1 \\\hline
		\end{tabular}}
	\end{table}
	
	\begin{table}
		\centerline{
			\begin{tabular}{|b{8mm}|b{8mm}|b{8mm}|}
				\hline
				x y z x y z x y z x y z x y z & x y z x y z x y z & 1 1 1 1 1 \\\hline
		\end{tabular}}
	\end{table}
	
	
	
	
	
	
	
	
	
\end{document}
