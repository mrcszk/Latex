\documentclass[a4paper,12pt]{article}
\usepackage[T1]{fontenc}
\usepackage[utf8]{inputenc}
\usepackage[polish]{babel}
\let\lll\undefined
\usepackage{amssymb}
\usepackage{amsthm}
\usepackage{times}
\usepackage{anysize}

\marginsize{1.5cm}{1.5cm}{1.5cm}{1.5cm}
\sloppy 

\theoremstyle{definition}
\newtheorem{df}{Definicja}


\begin{document}
Istnieje ścisły związek między rozkładem macierzy $A$ na macierze $L$ i $U$ a metodą eliminacji Gaussa. Można wykazać, że elementy kolejnych kolumn macierzy $L$ są równe współczynnikom przez które mnożone są w kolejnych krokach wiersze układu równań celem dokonania eliminacji niewiadomych w odpowiednich kolumnach. Natomiast macierz $U$ jest równa macierzy trójkątnej uzyskanej w eliminacji Gaussa
$$
	[A|b] =  \left[
	 \begin{array}{rrrr}
	  2 & 2 & 4 & 4\\
	  1 & 2 & 2 & 4\\
	  1 & 4 & 1 & 1\\
	 \end{array}
	 \right] 
	 =
	 \left[
	 \begin{array}{rrrr}
	 2 & 2 & 4 & 4\\
	 0 & 1 & 0 & 2\\
	 0 & 3 & -1 & -1\\
	 \end{array}
	 \right] 
	 =
	 \left[
	 \begin{array}{rrrr}
	 2 & 2 & 4 & 4\\
	 0 & 1 & 0 & 2\\
	 0 & 0 & -1 & -7\\
	 \end{array}
	 \right] 
$$
$$
	L = 
	\left[
	\begin{array}{rrr}
	1 & 0 & 0\\
	\frac{1}{2} & 1 & 0\\
	\frac{1}{2} & 3 & 1\\
	\end{array}
	\right] 	 
	 \qquad
	 U =
	 \left[
	 \begin{array}{rrrr}
	 2 & 2 & 4 & 4\\
	 0 & 1 & 0 & 2\\
	 0 & 0 & -1 & -7\\
	 \end{array}
	 \right] 	 
$$
\end{document}
