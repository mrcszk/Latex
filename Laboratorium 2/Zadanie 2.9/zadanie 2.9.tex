\documentclass[a4paper,12pt]{article}
\usepackage[T1]{fontenc}
\usepackage[utf8]{inputenc}
\usepackage[polish]{babel}
\let\lll\undefined
\usepackage{amssymb}
\usepackage{amsthm}
\usepackage{times}
\usepackage{anysize}

\begin{document}
	\textbf{Przykład 8.3.} Wykażemy, że funkcje
	$$
	f(x) = -\arctan x\qquad \textsl{i}  \qquad g(x) = \arccos \frac{x}{\sqrt{1+x^2}}$$
	różnią się jedynie o stałą $B = -\frac{\pi}{2}$.
	
	Dla każdego $x \in \mathbb{R}$, mamy:
	$$
	f'(x) = \frac{-1}{1+x^2},
	$$
	$$
	g'(x) = \frac{-1}{\sqrt{1-\left ( \frac{x}{\sqrt{1+ x^2}}
	\right )^2 }} \cdot
	\frac{\sqrt{1+x^2}-\frac{2x^2}{2\sqrt{1+x^2}}}{1+x^2}
	=
	\frac{-1}{1+x^2};
	$$
	oznacza to, że: 
	$$f'(x) = g'(x),$$
	więc na podstawie ostatniego wniosku możemy napisać:
	$$\forall x \in \mathbb{R}: f(x) = g(x) + B.$$
	Jednocześnie, np. dla x=0, mamy:
	$$f(0) = 0, \qquad g(0)=\frac{\pi}{2},$$
	zatem nietrudna zauważyć, że ostatnia równość ma miejsce, gdy $B = - \frac{\pi}{2}._{\blacksquare}$
\end{document}